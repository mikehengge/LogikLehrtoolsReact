%%%%%%%%%%%%%%%%%%%%%%%%%%%%%%%%%%%%%%%%%%%%%%%%%%%%%%%%%%%%%%%%%%%%
% Einleitung
%%%%%%%%%%%%%%%%%%%%%%%%%%%%%%%%%%%%%%%%%%%%%%%%%%%%%%%%%%%%%%%%%%%%

\chapter{Einleitung}\label{Einleitung}
In den letzten Jahrzehnten hat sich das Internet über die ganze Welt verbreitet. Aktuell dient das Internet vielerorts als primäres Informations- und Kommunikationsmedium. Dabei wandelten sich die Ansprüche an dieses Medium mit dem technologischen Fortschritt. Mit leistungsfähigerer Hardware, sowohl auf Nutzer als auch auf Entwicklerseite, und schnelleren Internetverbindungen wurden neue Inhalte und neue Wege der Darstellung möglich.\\
Dank dieser neuen Möglichkeiten werden mehr und mehr Anwendungen über das Web angeboten, welche vorher nur als Desktop Applikationen zur Verfügung standen. Um solche Webapplikationen zu entwickeln werden oftmals spezielle JavaScript-Bibliotheken und -Frameworks verwendet. Beispiele sind Angular, Vue, React noch viele mehr. React zeichnet sich dabei derzeit durch das größte Wachstum aus.\\
React ist eine open-source Java Bibliothek und wurde von Facebook entwickelt um möglichst performante Oberflächen zu gestalten. Seine Beliebtheit gründet sich auf der einfachen und schnellen Nutzung, mit hochperformanten Ergebnissen. Dabei beschränkt sich React auf den View-Teil der klassischen Model-View-Controller Struktur. Diese View besteht aus einzelnen Komponenten von denen jeder seinen eigenen 'State' verwaltet. Ändert sich diese States werden nur die betroffenen Komponenten neu gerendert, einer der Gründe für Reacts hohe Effizienz.\\\\
Die vorliegende Arbeit erfüllt zwei Funktionen und ist daher auch inhaltlich zweigeteilt. Zum einen sind in ihr Ergebnisse der Recherche und Evaluation von React aufgeführt, während sie sich zum anderen mit der Umsetzung des im Folgenden beschriebenen Projekts beschäftigt. \\
Ziel des Recherche-Teils ist es, Reacts Eigenschaften, Vorteile, Nachteile und generelle Nutzbarkeit darzustellen. Dieser Teil soll dem Anspruch gerecht werden, eine fundierte Entscheidung für oder gegen eine eigene Nutzung von React zu ermöglichen.\\\\
Bei dem Projekt handelt es sich um die Erstellung einer web-basierten Anwendung mittels JavaScript und React aus einer bestehenden stand-alone Java Implementation „Logik Lehrtools“. In dieser werden in einer Nutzeroberfläche die Algorithmen Resolution, Backward Dual Resolution und Variablen-Elimination aus der Aussagenlogik auf dort eingegebene Formeln angewandt und die Zwischenergebnisse und Endergebnisse der Algorithmen ausgegeben. \\
Das entstehende Online-Tool soll für die Lehre einsetzbar sein und eine Möglichkeit für Studenten bieten Beispiele selbst nachzuvollziehen, den Ablauf der einzelnen Algorithmen zu beobachten und eigene Lösungen zu kontrollieren. Hierfür wird es nach Fertigstellung über einen Webserver im Universitätsnetzwerk zur Verfügung gestellt. 


